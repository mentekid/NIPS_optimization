\documentclass{article}
\usepackage{amsmath}
\usepackage{amsthm}

% if you need to pass options to natbib, use, e.g.:
% \PassOptionsToPackage{numbers, compress}{natbib}
% before loading nips_2017
%
% to avoid loading the natbib package, add option nonatbib:
% \usepackage[nonatbib]{nips_2017}

\usepackage{nips_2017}

% to compile a camera-ready version, add the [final] option, e.g.:
% \usepackage[final]{nips_2017}

\usepackage[utf8]{inputenc} % allow utf-8 input
\usepackage[T1]{fontenc}    % use 8-bit T1 fonts
\usepackage{hyperref}       % hyperlinks
\usepackage{url}            % simple URL typesetting
\usepackage{booktabs}       % professional-quality tables
\usepackage{amsfonts}       % blackboard math symbols
\usepackage{nicefrac}       % compact symbols for 1/2, etc.
\usepackage{microtype}      % microtypography

% Could this be a better title?
\title{A generic and fast C++ optimization framework}

% The \author macro works with any number of authors. There are two
% commands used to separate the names and addresses of multiple
% authors: \And and \AND.
%
% Using \And between authors leaves it to LaTeX to determine where to
% break the lines. Using \AND forces a line break at that point. So,
% if LaTeX puts 3 of 4 authors names on the first line, and the last
% on the second line, try using \AND instead of \And before the third
% author name.

\author{
  Shikhar Bhardwaj \\
  \AND
  Yannis Mentekidis \\
  \AND
  Marcus Edel \\
  \AND
  Ryan R. Curtin \\
  Symantec Corporation \\
}

\begin{document}
% \nipsfinalcopy is no longer used

\maketitle

\begin{abstract}
The development of the {\bf mlpack} C++ machine learning library
(\url{http://www.mlpack.org/}) has required the design and implementation of a
flexible, robust optimization system that is able to solve the types of
arbitrary optimization problems that may arise all throughout machine learning
problems.  You care because:

\begin{itemize}
  \item It's easy to implement new optimizers.
  \item It's easy to implement new functions to be optimized.
  \item This system is more comprehensive than other techniques that may focus
only on neural networks or single problems.
  \item This system is also more specific and offers more room for optimization
than super general systems like scipy.optimize().
\end{itemize}

This abstract will need to be rewritten.
\end{abstract}

\section{Introduction}

In recent times, machine learning conferences have focused more and more on
optimization techniques.  For instance, this year at ICML (ICML 2017), every
single session had at least one track devoted solely to optimization
techniques.
<TODO: add some more filler here.  Show that optimization is an important part
of machine learning, in whatever ways.>

\begin{itemize}
  \item This sequence of events (or situation) motivates the development of a
generic, flexible, and fast optimization framework.
  \item We have implemented exactly this in {\bf mlpack}, and it's good.
\end{itemize}

{\bf <TODO: elaborate on above, add a few more sentences.>}

{\bf <TODO: mention related work, or really that we haven't found other
libraries that have a comprehensive infrastructure like the one we have built,
and that what we have built is more flexible and generic}

\section{mlpack: a scalable C++ machine learning library}

\begin{itemize}
  \item Some quick background on what mlpack is and what its goals are.
Probably the list of four goals is good here (though maybe not in itemized
format).
  \item Discuss why we've chosen C++, and how policy-based design gives us some
really nice flexibility.  (Maybe there is not fully space for that.)
\end{itemize}

\section{Requirements for optimizers and functions}

In general, we want to be able to consider any solver of the problem

\begin{equation}
\operatorname{argmin}_{x} f(x)
\end{equation}

\noindent for any function $f(x)$ that takes some vector input $x$.  However, it
is impossible to implement something both so generic and fast---for instance,
gradient-based approaches converge far more quickly than gradient-free
approaches (in general), so we must design an abstraction that is able to
simultaneously generalize to many problem types, as well as able to take
advantage of accelerations and optimizations.

Let us split the class of functions to be optimized into some subclasses:

\begin{itemize}
  \item {\bf arbitrary}: no assumptions can be made on $f(x)$
  \item {\bf differentiable}: $f(x)$ has a computable gradient $f'(x)$
  \item {\bf separable}: $f(x)$ is a sum of individual components: $f(x) =
\sum_{i} f_i(x)$
  \item {\bf categorical}: $x$ contains elements that can only take discrete
values
  \item {\bf numeric}: all elements of $x$ take values in $\mathcal{R}$
  \item {\bf sparse}: the gradient $f'(x)$ or $f_i(x)$ (for a separable
function) is sparse
  \item {\bf partially differentiable}: the gradient $f'_j(x)$ is computable for
individual elements $x_j$ of $x$
  \item {\bf bounded}: $x$ is limited in the values that it can take
\end{itemize}

Needless to say, it is impossible to create a system where every optimizer can
work with every possible type of function: a gradient-based optimizer cannot
reasonably be expected to work with an arbitrary function $f(x)$ where the
gradient is not computable or available.

Instead, the best we can hope to achieve is to maximize the flexibility
available, so that a user can easily implement a function $f(x)$ and have it
work with as many optimizers as possible.  For this, C++ policy-based design
aids us greatly: when implementing a function to be optimized, a user can
implement only a few methods and we can use C++ template metaprogramming to
check that the given functions match the requirements of the optimizer that is
being used.  When implementing an optimizer, a user can assume that the given
function to be optimized meets the required assumptions of the optimizers, and
encode those requirements.  Since we are using templates and C++, the resulting
code generated by the compiler can be identical to what a developer would write
if they were writing an optimizer specifically for the function $f(x)$ at
hand---this can provide significant efficiency gains.

{\bf <TODO: maybe we should add a table with optimizer types that work with
certain assumptions.  Maybe this can be a table, maybe a diagram, but some kind
of visual aid to split the space of objective function types would be nice.>}

\section{{\tt FunctionType} API}

In order to facilitate consistent implementations, we have defined a {\tt
FunctionType} API that describes all the functions that an objective function
may (or must) implement, depending on the subclass of objective function.

{\bf <TODO: we can add a list of each function and what it does.  A basic list
is below.  We can elaborate on this significantly.>}

\begin{itemize}
  \item {\tt Evaluate()}
  \item {\tt Gradient()}
  \item {\tt PartialGradient()} 
  \item ...
  \item constraints?
  \item discrete variable support?
  \item batch support
\end{itemize}

A user may implement as many (or as few) of these functions as they would like,
and this will allow them to use a larger set of optimizers.

{\bf <TODO: add very simple example function>}

\section{Static checking of functions}

{\bf <TODO: probably a better section title>}

Unfortunately, template metaprogramming can result in some very lengthy error
messages.  Therefore, we must be careful to ensure that a user is able to easily
debug a problem when they implement an objective function or gradient.  To
provide good error message output, we can use C++'s template metaprogramming
support to determine what methods a type has available.  Similarly, we can also
use static compile-time constants to denote the methods that are require by a
specific optimizer.  This is implemented via SFINAE \cite{sfinae}.  A {\tt
static\_assert()} is raised when a given objective function does not have the
required methods for a given optimizer.  {\bf <TODO: implement this, so we are
not lying.  But it should be easy enough, maybe an hour of work max.>}

For instance, when attempting to use the L-BFGS optimizer without having a
{\tt Gradient()} function implemented, the user will receive the following error
message:

{\bf <TODO: put error message here.  If we are short on space, let's cut this
example and make this section shorter.>}

\section{Supported optimizers and functions in mlpack}

Thanks to the easy abstraction, we have been able to provide support for a large
set of diverse optimizers and objective functions.  Below is a list of what is
currently available.

\begin{itemize}
  \item {\bf <TODO: add all optimizers>}
  \item {\bf <TODO: add all things that we can optimize>}
\end{itemize}

\section{Conclusion}

We have identified that the support for generic and robust optimization is not
currently available in most machine learning toolkits, and acted upon this
observation to provide an easy framework for both implementing new optimizers
and new objective functions to be optimized inside of the {\bf mlpack} machine
learning library.  This supports a 

\end{document}
